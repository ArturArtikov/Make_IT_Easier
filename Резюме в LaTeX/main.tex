\documentclass{article}
\usepackage[T2A]{fontenc}
\usepackage[english, russian]{babel}
\usepackage{graphicx}

\input{setup/preamble.sty}
\input{setup/macros.sty}

\begin{document}
    \begin{tcolorbox}[height=0.19\textheight,colframe=white,colback=white]
        \begin{minipage}{0.81\textwidth} 
            \vspace{3em}
            \LARGE{\name{Артур Артиков}{}}
            \vspace{1em}
            \Large{DATA SCIENTIST}
            \newline
            \newline
            \normalsize{\address{Москва, Россия}{}}
            \vspace{0.3em}
            \email{artikovartur@internet.ru}
            \vspace{0.3em}
            \phone{+7 965 161-18-88}
            \newline
            \faTelegram{ \href{https://t.me/ArturArtikov}{ ArturArtikov}}
            \faGithub{ \href{https://github.com/ArturArtikov/Portfolio}{ Data Science Portfolio}}
            \newline
        \end{minipage}
        \begin{minipage}{0.177\textwidth}
            \includegraphics[width=1\textwidth, trim={0 0.8cm 0cm 0}, clip]{photo.png} % Picture
        \end{minipage} \hfill
        \subsection*{}
    \end{tcolorbox}

    \begin{minipage}{0.7125\textwidth} % Main Panel (e.g. Education, Work Experience)
        \begin{tcolorbox}[height=0.8\textheight, grow to left by=0.55cm,colframe=white,colback=white]

            %%%%%%%%%%%% Опыт работы %%%%%%%%%%%%
            \subsection*{\Large{Опыт работы}}

            \entryDesc{02.2024 - н.в.}
            {Репетитор по программированию на Python (6-11 класс)}{в Онлайн-школе Фоксфорд}
            {\begin{compactitem}[-]
                \item Создание учебных программ для обучения программированию учеников 6-11 класса
                \item Проведение уроков на онлайн-платформе
                \item Создание и проверка домашних заданий, ревью кода
            \end{compactitem}
            }

            \entryDesc{03.2024 - н.в.}
            {Data Scientist}{участие в Хакатонах и Кейс-чемпионатах}
            {\begin{compactitem}[-]
                \item Проводил визуальный анализ данных с помощью библиотек Seaborn, Plotly
                \item Работал с датасетами с помощью Pandas, преобразовывал данные, проводил EDA               
                \item Создавал модели машинного обучения с помощью LightGBM, CatBoost, XGBoost
                \item Налаживал работу команды, организовывал рабочий процесс с помощью Trello, Google Calendar, Zoom
                \item Оформлял репозитории в GitHub, создавал презентации, проводил защиту проектов
            \end{compactitem}
            }

            \entryDesc{06.2023}
            {Backend Python Developer (Практика)}{в МАДИ}
            {\begin{compactitem}[-]
                \item Создал чат-бота в Telegram
                \item Прописал команды чат-бота с помощью библиотеки pyTelegramBotAPI
                \item Подключил к боту библиотеку TensorFlow, для генерации ответов на заданный пользователем вопрос
            \end{compactitem}
            }
            
            \subsection*{\Large{Сертификаты и Дипломы}}
            {Названия и активные ссылки:}
            {\begin{compactitem}[*]
                \item{\href{https://disk.yandex.ru/i/u70FDtYfRM2Xqg}{Alfa FinU Hack. Диплом финалиста}}
                \item{\href{https://disk.yandex.ru/i/zvw_n36lRkEEbQ}{Axenix Business Cup. Сертификат полуфиналиста}}
                \item{\href{https://disk.yandex.ru/i/OpMpHdrC_FpKSw}{Changellenge Cup IT. Диплом HQA}}
                \item{\href{https://disk.yandex.ru/i/iD6kRkfoQEhBUw}{Changellenge Cup Russia. Диплом участника}}
                \item{\href{https://disk.yandex.ru/i/PZdSphoZOyIMpg}{IT Purple Hack. Сертификат участника}}
                \item{\href{https://disk.yandex.ru/i/5ZDqyT28396w3g}{ML TalentMatch. Сертификат участника}}
                \item{\href{https://disk.yandex.ru/i/bd6KZTIkSCeq2w}{Мегафон х СПбГУ. Олимпиада для первых. Сертификат полуфиналиста}}
            \end{compactitem}
            }

            %%%%%%%%%%%% Образование %%%%%%%%%%%%
            \subsection*{\Large{Образование}}
            \large{\textbf{\vspace{1mm}Прикладная математика, Бакалавр}}\hfill\hfill
            \newline
            \small{МОСКОВСКИЙ АВТОМОБИЛЬНО-ДОРОЖНЫЙ ГОСУДАРСТВЕННЫЙ ТЕХНИЧЕСКИЙ УНИВЕРСИТЕТ (МАДИ)}

            \subsection*{\Large{Курсы}}
            {\begin{compactitem}[]
                \item{\normalsize{\textbf{SkillFactory}{ - специализация Data Science}}}
                \newline
                \item{\normalsize{\textbf{ЦБ РФ}{ - Финансовые технологии и инновации в платежах}}}
            \end{compactitem}
            }

        \end{tcolorbox}
    \end{minipage}
    \begin{minipage}{0.23\textwidth} % Side Panel (e.g. Skills, Links, Languages, etc.)
        \begin{tcolorbox}[height=0.8\textheight, grow to right by=0.5cm, colback=white,colframe=white,arc=1mm]
            % Skills, the skill level is drawn as bars, input: skill name and an array starting from 0 and ending before 4
                
            \subsection*{\Large{Языки}}
                \skill{Русский C1}{5}
                \skill{Немецкий B2}{4}
                \skill{Английский B1}{3}

            \subsection*{\Large{Навыки}}
                \skill{Python}{5}
                \skill{PostgreSQL}{4}
                \skill{Pandas}{5}
                \skill{NumPy}{5}
                \skill{Matplotlib}{5}
                \skill{Seaborn}{5}
                \skill{Plotly}{4}
                \skill{SciPy}{4}
                \skill{Statsmodels}{4}
                \skill{Sklearn}{5}
                \skill{CatBoost}{4}
                
        \end{tcolorbox}
    \end{minipage}

\end{document}
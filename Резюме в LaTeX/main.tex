\documentclass{article}
\usepackage[T2A]{fontenc}
\usepackage[english, russian]{babel}
\usepackage{graphicx}

%%%%%%%%%%%%%%%%%%%%%%%%%%%%%%%%%%%%%%%% IMPORTS %%%%%%%%%%%%%%%%%%%%%%%%%%%%%%%%%%%%%%%%
\documentclass[10pt,onesize,a4paper,titlepage]{article}

%%%%%%%%%%%%%%% Formatting %%%%%%%%%%%%%%% 
\usepackage[english]{babel}
\usepackage[utf8]{inputenc}
\usepackage{geometry} % Margins
\usepackage{sectsty} % Custom Sections
\usepackage{paralist} % For compactitem

%%%%%%%%%%%%%%% Font %%%%%%%%%%%%%%% 
%\usepackage{Archivo}
\usepackage[T1]{fontenc} 
\renewcommand{\familydefault}{\sfdefault} % Set Helvetica default

%%%%%%%%%%%%%%% Graphics %%%%%%%%%%%%%%% 
\usepackage{fontawesome5} % Icons
\usepackage{graphicx} % Images
\usepackage[most]{tcolorbox} % Color Box
\usepackage{xcolor} % Colors
\usepackage{tikz} % For Drawing Shapes
\tcbuselibrary{breakable}

%%%%%%%%%%%%%%% Miscelanous %%%%%%%%%%%%%%% 
\usepackage{lipsum} % Lorem Ipsum
\usepackage{hyperref} % For Hyperlinks

%%%%%%%%%%%%%%% Colors %%%%%%%%%%%%%%% 
%\definecolor{accent}{HTML}{9fbfc0} % Accent color
\definecolor{accent}{HTML}{789ABE} % Accent color
\definecolor{lightgrey}{HTML}{f2f2f2} % Title color
\definecolor{skillbar}{HTML}{ffffff} % Empty skill bars

%%%%%%%%%%%%%%% Section Format %%%%%%%%%%%%%%% 

% Change font of \section command
\sectionfont{
    \Large
    \fontseries{m}\selectfont 
    \sectionrule{0pt}{0pt}{-8pt}{1pt}
}

\subsectionfont{
    \large
    \fontseries{m}\selectfont 
    \sectionrule{0pt}{0pt}{-8pt}{1pt}
}

%%%%%%%%%%%%%%% Margins and Headers %%%%%%%%%%%%%%%
\geometry{
  a4paper,
  left=7mm,
  right=7mm,
  bottom=10mm,
  top=10mm
}

\pagestyle{empty} % Empty Headers


%%%%%%%%%%%%%%%%%%%%%%%%%%%%%%%%%%%%%%%% MACROS %%%%%%%%%%%%%%%%%%%%%%%%%%%%%%%%%%%%%%%%

%%%%%%%%%%%%%%% Link With an Icon %%%%%%%%%%%%%%% 
\newcommand{\link}[1]{
    \href{#1}{\faIcon{link}}
}

%%%%%%%%%%%%%%% Name Template %%%%%%%%%%%%%%% 
\newcommand{\name}[2]{
    % Name
    \Huge % Font size
    \raggedright \textbf{#1} \par

    \vspace*{0.3cm}
    
    % Profession
    \Large % Font size
    \raggedright #2 \par
    \normalsize \normalfont
}

%%%%%%%%%%%%%%% Contact Details %%%%%%%%%%%%%%%
\newcommand{\info}[2]{
    {\color{accent}\faIcon{#2}} \hspace{0.2em} #1
}

%%%%%%%%%%%%%%% Email %%%%%%%%%%%%%%%
\newcommand{\email}[1]{
    \info{#1}{envelope}
}

%%%%%%%%%%%%%%% Phone Number %%%%%%%%%%%%%%%
\newcommand{\phone}[1]{
    \info{#1}{mobile-alt}
}

%%%%%%%%%%%%%%% Address %%%%%%%%%%%%%%%
\newcommand{\address}[1]{
    \info{#1}{map-marker-alt}
}

%%%%%%%%%%%%%%% GitHub %%%%%%%%%%%%%%%
\newcommand{\github}[2]{
    \info{\href{#1}{\underline{#2}}}{github}
}

%%%%%%%%%%%%%%% LinkedIn %%%%%%%%%%%%%%%
\newcommand{\linkedin}[2]{
    \info{\href{#1}{\underline{#2}}}{linkedin}
}

%%%%%%%%%%%%%%% Website %%%%%%%%%%%%%%%
\newcommand{\website}[1]{
    \info{#1}{link}
}



%%%%%%%%%%%%%%%%%%%%%%%%%%%%%%%%%%%%%%%%%%%%%%%%%%%%%%%%%%%%%%%%%%%%%%%%%%%%%%%%
%%%%%%%%%%%%%%%%%%%%%%%%%%%%%% ORIGINAL %%%%%%%%%%%%%%%%%%%%%%%%%%%%%%%%%%%%%%%%
%%%%%%%%%%%%%%%%%%%%%%%%%%%%%%%%%%%%%%%%%%%%%%%%%%%%%%%%%%%%%%%%%%%%%%%%%%%%%%%%


%%%%%%%%%%%%%%% Education %%%%%%%%%%%%%%%
\newcommand{\education}[4]{
    % Name of the studies
    \noindent \large \parbox{.7\linewidth}{\textbf{#1}}
    % Duration in a Box
    \hfill \scriptsize
    \tcbox[enhanced,box align=base,nobeforeafter,colback=title,colframe=title,size=fbox,arc=0mm]{\textbf{#2}} \par
    \vspace{0.3em}
    % School Name 
    \large
    \noindent \color{title} \parbox{.7\linewidth}{\textsl{#3}} \par
    % Description
    \normalsize \color{black}
    \vspace*{0.3em}
    \small #4 
    \normalsize \par
}

%%%%%%%%%%%%%%% Work Experience %%%%%%%%%%%%%%%
\newcommand{\work}[4]{
    % Name of the Job
    \noindent \large \parbox{.7\linewidth}{\textbf{#1}}
    % Duration in a Box 
    \hfill \scriptsize
    \tcbox[enhanced,box align=base,nobeforeafter,colback=title,colframe=title,size=fbox,arc=0mm]{\textbf{#2}} \par
    \vspace{0.3em}
    % Name of the Employer
    \noindent \large \color{title} \parbox{.7\linewidth}{\textsl{#3}} \par
    % Description of the job
    \vspace*{0.3em} \color{black}
    \small #4 
    \normalsize \par
}

%%%%%%%%%%%%%%% Publications %%%%%%%%%%%%%%%
\newcommand{\pub}[5]{
    % Title
    \noindent \large \parbox{.7\linewidth}{\textbf{#1} \link{#5}}
    % Publication Date
    \hfill \scriptsize
    \tcbox[enhanced,box align=base,nobeforeafter,colback=title,colframe=title,size=fbox,arc=0mm]{\textbf{#2}} \par
    \vspace{0.3em}
    % Institution
    \large
    \noindent \color{title} \parbox{.7\linewidth}{\textsl{#3}} \par
    % Description
    \vspace*{0.3em} \color{black}
    \small \textit{#4} \par
    \normalsize \par 
}

%%%%%%%%%%%%%%% Draw Skill Bars %%%%%%%%%%%%%%% 
%\newcommand{\drawskillbars}[1]{
%    \begin{tikzpicture}
%        % Draw 5 gray bars
%        \foreach \i in {0, 1, 2, 3, 4}{
%            \fill[lightgray] (\i * 0.7 + 0.2 *\i,0) rectangle (0.7 + \i * 0.7 + \i * 0.2,0.1);
%        }
%        
%        % Draw number of black bars depending on the skill level
%        \foreach \i in {#1}{
%            \fill[black] (\i * 0.7 + 0.2 *\i,0) rectangle (0.7 + \i * 0.7 + \i * 0.2,0.1);
%        }
%    \end{tikzpicture} \par
%}
    
%%%%%%%%%%%%%%% Skills %%%%%%%%%%%%%%%
%\newcommand{\skill}[2]{
%    % Name of the skill
%    \large
%    \noindent \hangafter=0
%    \textmd{#1}
%    \normalsize \par 
%    % Skill bars
%    \drawskillbars{#2}
%    \vspace{1.5em}
%}


%%%%%%%%%%%%%%% Language %%%%%%%%%%%%%%%
\newcommand{\lan}[2]{
    % Name of the language
    \large
    \noindent \hangafter=0
    \textmd{#1}
    % Knowledge level
    \drawskillbars{#2}
    \vspace{1em}
 }


%%%%%%%%%%%%%%%%%%%%%%%%%%%%%%%%%%%%%%%%%%%%%%%%%%%%%%%%%%%%%%%%%%%%%%%%%%%%%%%%
%%%%%%%%%%%%%%%%%%%%%%%%%%%%%% CUSTOM %%%%%%%%%%%%%%%%%%%%%%%%%%%%%%%%%%%%%%%%%%
%%%%%%%%%%%%%%%%%%%%%%%%%%%%%%%%%%%%%%%%%%%%%%%%%%%%%%%%%%%%%%%%%%%%%%%%%%%%%%%%


\newcommand{\mySection}[1]{
    \Large\uppercase{#1} \\[-1.5ex]
    {\color{accent}\rule{\linewidth}{2pt}} \par
}

%%%%%%%%%%%%%%% Entry without Date %%%%%%%%%%%%%%%
\newcommand{\entryNoDate}[1]{
    \noindent\parbox{0.9\linewidth}{\normalsize #1} \par
    \vspace*{1em}
}

%%%%%%%%%%%%%%% Entry with Date %%%%%%%%%%%%%%%
\newcommand{\entry}[2]{
    % What
    \noindent\parbox{0.76\textwidth}{\large\textbf{#2}}
    % Time
    \parbox{0.23\textwidth}{\hfill\normalsize\textit{#1}} \par
    \vspace*{1em}
}

%%%%%%%%%%%%%%% Entry with Focus %%%%%%%%%%%%%%%
\newcommand{\entryFocus}[3]{
    % What
    \noindent\parbox{0.76\textwidth}{\large\textbf{#2}}
    % Time
    \parbox{0.23\textwidth}{\hfill\normalsize\textit{#1}} \par
    % Average, Focus, etc
    \noindent\parbox{0.9\linewidth}{\normalsize\textit{#3}} \par 
    \vspace*{1em}
}

%%%%%%%%%%%%%%% Entry with Focus and Description %%%%%%%%%%%%%%%
\newcommand{\entryFocusDesc}[5]{
    % What
    \noindent\parbox{0.76\textwidth}{\large\textbf{#2}
    % Where
    \normalsize\text{#3}}
    % Time
    \parbox{0.23\textwidth}{\hfill\normalsize\textit{#1}} \par
    \vspace{-0.2em}
    % Average, Focus, etc
    \noindent\parbox{0.9\linewidth}{\normalsize\textit{#4}} \par
    \vspace*{0.3em}
    % Description
    \noindent\hangindent=0.5cm\hangafter=0\parbox{0.9\linewidth}{\normalsize{#5}}
    \vspace*{1em}
}

%%%%%%%%%%%%%%% Entry with Description %%%%%%%%%%%%%%%
\newcommand{\entryDesc}[4]{
    % What
    \noindent\parbox{0.76\textwidth}{\large\textbf{#2}
    % Where
    \normalsize\text{#3}}
    % Time
    \parbox{0.23\textwidth}{\hfill\normalsize\textit{#1}} \par
    \vspace*{0.3em}
     % Description
    \noindent\hangindent=0.5cm\hangafter=0 \parbox[inner-position=]{0.9\linewidth}{\normalsize#4}
    \vspace*{1em}
}


%%%%%%%%%%%%%%% Draw Skill Bars %%%%%%%%%%%%%%% 
\newcommand{\drawskillbars}[1]{
    \begin{tikzpicture}
        % Draw 5 gray bars
        \foreach \i in {0, 1, 2, 3, 4}{
         \pgfmathparse{(\i<=#1-1?("black"):"skillbar"}
        \edef\bulletcolor{\pgfmathresult}
            \fill[\bulletcolor] (\i * 0.7 + 0.1 *\i,0) rectangle (0.7 + \i * 0.7 + \i * 0.1,0.1);
        }
    \end{tikzpicture} \par
}
    
%%%%%%%%%%%%%%% Skills %%%%%%%%%%%%%%%
\newcommand{\skill}[2]{
    % Name of the skill
    \noindent \hangafter=0\textmd{#1}
    \normalsize \par 
    % Skill bars
    \drawskillbars{#2}
    \vspace{1em}
}


\begin{document}
    \begin{tcolorbox}[height=0.19\textheight,colframe=white,colback=white]
        \begin{minipage}{0.81\textwidth} 
            \vspace{3em}
            \LARGE{\name{Артур Артиков}{}}
            \vspace{1em}
            \Large{DATA SCIENTIST}
            \newline
            \newline
            \normalsize{\address{Москва, Россия}{}}
            \vspace{0.3em}
            \email{artikovartur@internet.ru}
            \vspace{0.3em}
            \phone{+7 965 161-18-88}
            \newline
            \faTelegram{ \href{https://t.me/ArturArtikov}{ ArturArtikov}}
            \faGithub{ \href{https://github.com/ArturArtikov/Portfolio}{ Data Science Portfolio}}
            \newline
        \end{minipage}
        \begin{minipage}{0.177\textwidth}
            \includegraphics[width=1\textwidth, trim={0 0.8cm 0cm 0}, clip]{photo.png} % Picture
        \end{minipage} \hfill
        \subsection*{}
    \end{tcolorbox}

    \begin{minipage}{0.7125\textwidth} % Main Panel (e.g. Education, Work Experience)
        \begin{tcolorbox}[height=0.8\textheight, grow to left by=0.55cm,colframe=white,colback=white]

            %%%%%%%%%%%% Опыт работы %%%%%%%%%%%%
            \subsection*{\Large{Опыт работы}}

            \entryDesc{02.2024 - н.в.}
            {Репетитор по программированию на Python (6-11 класс)}{в Онлайн-школе Фоксфорд}
            {\begin{compactitem}[-]
                \item Создание учебных программ для обучения программированию учеников 6-11 класса
                \item Проведение уроков на онлайн-платформе
                \item Создание и проверка домашних заданий, ревью кода
            \end{compactitem}
            }

            \entryDesc{03.2024 - н.в.}
            {Data Scientist}{участие в Хакатонах и Кейс-чемпионатах}
            {\begin{compactitem}[-]
                \item Проводил визуальный анализ данных с помощью библиотек Seaborn, Plotly
                \item Работал с датасетами с помощью Pandas, преобразовывал данные, проводил EDA               
                \item Создавал модели машинного обучения с помощью LightGBM, CatBoost, XGBoost
                \item Налаживал работу команды, организовывал рабочий процесс с помощью Trello, Google Calendar, Zoom
                \item Оформлял репозитории в GitHub, создавал презентации, проводил защиту проектов
            \end{compactitem}
            }

            \entryDesc{06.2023}
            {Backend Python Developer (Практика)}{в МАДИ}
            {\begin{compactitem}[-]
                \item Создал чат-бота в Telegram
                \item Прописал команды чат-бота с помощью библиотеки pyTelegramBotAPI
                \item Подключил к боту библиотеку TensorFlow, для генерации ответов на заданный пользователем вопрос
            \end{compactitem}
            }
            
            \subsection*{\Large{Сертификаты и Дипломы}}
            {Названия и активные ссылки:}
            {\begin{compactitem}[*]
                \item{\href{https://disk.yandex.ru/i/u70FDtYfRM2Xqg}{Alfa FinU Hack. Диплом финалиста}}
                \item{\href{https://disk.yandex.ru/i/zvw_n36lRkEEbQ}{Axenix Business Cup. Сертификат полуфиналиста}}
                \item{\href{https://disk.yandex.ru/i/OpMpHdrC_FpKSw}{Changellenge Cup IT. Диплом HQA}}
                \item{\href{https://disk.yandex.ru/i/iD6kRkfoQEhBUw}{Changellenge Cup Russia. Диплом участника}}
                \item{\href{https://disk.yandex.ru/i/PZdSphoZOyIMpg}{IT Purple Hack. Сертификат участника}}
                \item{\href{https://disk.yandex.ru/i/5ZDqyT28396w3g}{ML TalentMatch. Сертификат участника}}
                \item{\href{https://disk.yandex.ru/i/bd6KZTIkSCeq2w}{Мегафон х СПбГУ. Олимпиада для первых. Сертификат полуфиналиста}}
            \end{compactitem}
            }

            %%%%%%%%%%%% Образование %%%%%%%%%%%%
            \subsection*{\Large{Образование}}
            \large{\textbf{\vspace{1mm}Прикладная математика, Бакалавр}}\hfill\hfill
            \newline
            \small{МОСКОВСКИЙ АВТОМОБИЛЬНО-ДОРОЖНЫЙ ГОСУДАРСТВЕННЫЙ ТЕХНИЧЕСКИЙ УНИВЕРСИТЕТ (МАДИ)}

            \subsection*{\Large{Курсы}}
            {\begin{compactitem}[]
                \item{\normalsize{\textbf{SkillFactory}{ - специализация Data Science}}}
                \newline
                \item{\normalsize{\textbf{ЦБ РФ}{ - Финансовые технологии и инновации в платежах}}}
            \end{compactitem}
            }

        \end{tcolorbox}
    \end{minipage}
    \begin{minipage}{0.23\textwidth} % Side Panel (e.g. Skills, Links, Languages, etc.)
        \begin{tcolorbox}[height=0.8\textheight, grow to right by=0.5cm, colback=white,colframe=white,arc=1mm]
            % Skills, the skill level is drawn as bars, input: skill name and an array starting from 0 and ending before 4
                
            \subsection*{\Large{Языки}}
                \skill{Русский C1}{5}
                \skill{Немецкий B2}{4}
                \skill{Английский B1}{3}

            \subsection*{\Large{Навыки}}
                \skill{Python}{5}
                \skill{PostgreSQL}{4}
                \skill{Pandas}{5}
                \skill{NumPy}{5}
                \skill{Matplotlib}{5}
                \skill{Seaborn}{5}
                \skill{Plotly}{4}
                \skill{SciPy}{4}
                \skill{Statsmodels}{4}
                \skill{Sklearn}{5}
                \skill{CatBoost}{4}
                
        \end{tcolorbox}
    \end{minipage}

\end{document}